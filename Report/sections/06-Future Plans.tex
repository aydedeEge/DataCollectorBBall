%------------------------------------------------------------------
% Plan for Next Semester Section:
% In this section you should describe a plan for next semester. What tasks remain to be done, and what is the timeline for accomplishing these tasks? Furthermore you can describe how you plan to test your design in order to make sure it meets the desired specifications. Also outline the design decisions that are required in order to make sure that the final product can be tested.
%------------------------------------------------------------------

The design phase and data collection phase are mostly complete, but there is still a significant amount of building and testing that need to be completed before we have a full solution. What still needs to be done can be seen below.

\subsection{Improved Neural Net}
%TODO cross validation citation
As mentioned in our Design and Results section, our current neural network takes in as input the scores of the players who are playing in a match. It then outputs which team it thinks will win the match, with a probability attached to the output. Although this was the final solution of this term, it is still not sufficient to solve our problem. We need to be able to output "game scores" for each player, instead of which team wins. The final neural network should take in the player scores, and output the player game scores. The distinction is that the game scores are related to how well the net thinks that the player will actually play in that game, which is ultimately all that matters for the fantasy competitions. Once we have these game scores, we will then pass these to the third system, our lineup chooser.
Another improvement we can make with respect to the neural network has to do with the NN parameters. These parameters include the learning rate, the amount of hidden layers, the amount of nodes in each hidden layer, and the amount of data used. There is a technique called "n-fold cross-validation" which we plan to use to optimize these parameters. It involves testing different sets of values of these parameters to determine the values that yield the highest success rate.

\subsection{System 3: Lineup Chooser}
%TODO add figure below.
Currently, we do not have a system that chooses lineups out of the players playing. After modifying the neural network to output the game scores, we will have a list of players ordered by how well we think they are going to perform tonight. With no constraints, we could simply take the highest scoring players. However, as mentioned in the Requirements section, fantasy basketball imposes constraints on the lineup. The first constraint is the budget of the lineup. Each player will have a cost, and our system will have to be able to generate a high-scoring lineup while keeping the total cost of the players under the total budget. The next constraint is that we have to select players with specific basketball positions. Thus, the system will also have to take in these positions and be able to generate lineups with it. Finally, as there will likely be many possible lineups with similar estimated scores, the system will need to have a random aspect to it that allows it to generate different lineups. % A block diagram of this system can be seen in Figure X. 
This system will likely initially be implemented as a simple algorithm, but may have to be adapted, and could even end up being implemented as another neural network.

\subsection{Player Playtime}
One important statistic that we are missing from our system is the player playtime, i.e. how much the player plays each game. This is clearly an important factor in determining his score, since even if a player has a high per-unit-time score (e.g. 5 free throws per minute), if that player only plays for a minute, his score will be low.
Although we have this data in the database, we do not use it. The reason for not using it is that it cannot (currently) be used as an input for the neural network, since we do not have a way to predict how long a player will play for in a game in the future. That is to say, we would not be able to use our neural network for an upcoming game, as we would have missing data. There are two approaches to remedy this. The first is to create a system that estimates the playtime for a player for a future game. This system could either be a simple algorithm, like "average the playtime that player has received over the past n games". Alternatively, an argument could be made that this system could be implemented with some sort of predictive machine learning. The second approach would be to use the playtimes in calculating the player's career score. Instead of adding playtime as an input to the neural net on a per-match basis, we would change the players' scores that are used as inputs for the neural net to be multiplied by how much they played last season, or during the last n matches, on average. This is perhaps too gross a simplification, since the playtime may change on a more match-per-match basis, so both of these approaches will have to be explored.

\subsection{Testing}
Once the above tasks have been completed, we will be at the testing stage of our neural network. We will first test it for games in the past, and see how our lineups would have done in contrived fantasy competitions. The testing process is detailed as follows:

%TODO testing process
If, after performing our tests, we see that there is an adequate (i.e. profitable) success rate, we will move to test it on real fantasy competitions. However, if we see that there is not an adequate success rate, or if we want to improve the already adequate success rate, we will have to make adjustments. We already have many ideas for adjustments that we can make, including the following:
\begin{itemize}
\item \textbf{Improve player scores system (system 1):} Currently, our first system takes a player's statistics and outputs a score representing how good that player is, "in a vacuum" (i.e. not against any particular team). The current equation we use for this is just the fantasy scoring equation. Although there is certainly a correlation between fantasy score, and how good a player is, it is likely not the most accurate relationship. For one, the fantasy score does not take advantage of all of the information we have, such as the player's age, shot accuracy, the player's win-loss record, and the player's plus-minus, which is an indicator of how well a team does as a whole, when the player is on the court. The new system could either take all of these factors into account, or could outsource to use someone else's equation/algorithm. An argument for making the latter decision is that we do not know more about basketball than the experts do, and they can likely come up with better numerical representations of how good a player is than we can. One argument against this decision is that the information is hard to obtain, and in order to train the neural network on past games, we would need to know the details of the algorithm to be able to calculate the score for these older games (as opposed to just getting the output).
\item \textbf{Improve player playtime:} As mentioned in the Player Playtime subsection above, our initial solution to playtime will likely be to make the assumption that a player's playtime during the previous year is a good estimation of their future playtime. We also mention that a better solution could be designed that takes advantage of the playtime per game data. Moving to this second solution could help improve results.
\item \textbf{Obtain more accurate positional data:} Currently, our source, NBA stats, gives us three posible positions: guard (G), center (C), and forward (F). The positions required for fantasy competitions are more specific, such as shooting guard (SG), or small forward (SF). Collecting this more precise positional data could lead to a higher success rate.
\end{itemize}
%TODO add timeline
%\subsection{Timeline}

% There will have to be a timeline here