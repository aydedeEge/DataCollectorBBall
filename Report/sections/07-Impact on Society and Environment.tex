%\textit{Whut. Only software product, nothing in the real world. Can be used to predict which player are best to add your team if system successful (If system can predict performance, it will mainly be because it was able to determine the correlation between the players). Need high processing power as system gets more complex.}

\subsection{Use of Non-Renewable Resources}
Seeing as our entire project is made up of software components, the environmental impact, in particular the use of non-renewable resources, is fairly minimal. Having said this, the training aspect of the neural network certainly requires processing power, which consumes energy. The amount of energy consumed by a CPU or GPU varies greatly, but, for high end hardware, the total power usage can reach 500 W \cite{GPU}. As the amount of data the net is trained on increases, and the amount of times we retrain the net increases, so will the amount of time the computer runs for, and thus so will the total energy consumed. The amount of energy used by our system, and thus the total environmental impact, will be directly proportional to the amount of times the network is trained by all users. Theoretically, if this system was to be released to the public, then the accumulation of everyone training their respective neural networks could eventually become an environmental concern. With one thousand users training a net for an average of ten hours, we would see an energy consumption of about 18 GJ, which is equivalent to charging one million iPhones \cite{iphone}. Although this may seem insignificant, with one million users, the energy used would be equivalent to charging one billion iPhones. Currently, since we are in Quebec, the energy being consumed is almost entirely made up of that generated with renewable resources. However, most of the planet's energy is produced via non-renewable resources\cite{energy}. Thus, if the project ends up growing to this scale, it would be important to continue to consider and evaluate the use of non-renewable resources.

\subsection{Environmental Benefits}
If our technology was developed to the point at which it could be used with relatively low power requirements (e.g. no need for each user to train the neural network), then widespread use and adoption of our software could potentially reduce the amount of energy consumed by the fantasy sports community. Curretntly, fantasy sports participants spend significant amounts of time online trying to figure out how to improve their fantasy teams; some spend several hours a day doing such research \cite{fantasySportsTimeSpent}. If our system was adopted by a large enough portion of the people playing fantasy sports, and could be run by their computers easily, it could remove all the research and decision making from the hands of these participants. This could then in turn mean they will spend less time on their computers managing their fantasy teams, thus using less energy. This is a somewhat special case, as we would have to develop a version of our product that comes pre-trained, runs with low power, and is distributed in such a way that the user does not use more energy downloading and setting up the software than he/she saves with it. Currently, this is out of the scope of the project.

\subsection{Safety and Risk}
The most significant risk associated with our project will actually be posed to the fantasy sports industry itself. If we are able to successfully use machine learning to predict the optimal line-ups for fantasy sports, then the industry may switch from a human-human competition to an entirely ML one. After releasing the software, there could also be a high amount of people who were not previously playing fantasy sports who may now see it as profitable. These people may flood in and take over the game. After a while, however, it would no longer be profitable, since everyone playing would have the best, and equivalent, equipment. As well, more traditional fans may argue that it removes the fun or skill of fantasy sports, and may stop playing. If the project turns out to be exceptionally successful, and is released to the public, it could completely change the way that fantasy sports is played, and we see this as a risk to that industry. However, we also feel that it is unlikely that the system is able to outperform all humans, and is certainly unlikely that it would be adopted in such a way as to ruin the industry.

\subsection{Benefits to Society}
There are a few benefits to society that could result from this project, if it is successful. The first benefit of our project would be to hopefully help people become interested in neural networks and machine learning in general. We argue that this is important since more and more devices use machine learning techniques, and this would lead to a better understanding of the current technology. The second benefit is that it could help further the movement of data-backed sports hiring decisions, which is to use statistics for selecting players, rather than intuition. Michael Lewis writes about this method, and its success, in his book Moneyball \cite{moneyball}. The reason this is a benefit to society is because it could make the hiring process for sport players more blind, and more friendly towards minorities. The third benefit is related to expanding the list of domains to which machine learning techniques can be applied. If we are able to demonstrate a successful system for predicting optimal fantasy lineups, then others may try to apply a similar technique to a different problem. The problem could be something more meaningful, such as a pharmaceutical company determining how to select an optimal drug, which is an example mentioned in Picking Winners, by David Hunter et al \cite{picking_winners}.