%------------------------------------------------------------------------
% Introduction/Motivation/Objective Section: 
% In this section you should introduce (at a high level) the overall theme of the project, and state clearly what are the goals you are trying to achieve. This section should also clearly convey why this project is important, what is the potential impact (applications, etc).
%------------------------------------------------------------------------

The primary objective of this project is to build a machine learning system to predict winning line-ups of basketball players for daily fantasy basketball competitions. A winning line-up is any line-up that, upon being entered into the competition, returns more money than the entry fee (gains a profit). We hypothesized this could be a successful machine learning project since we believe that there are unknown relationships between players in a match, both on the same team and on opposing teams. The project was separated into milestones:
\begin{enumerate}
\item{Learn the necessary material regarding topics in machine learning}
\item{Design a system architecture}
\item{Collect a sufficient amount of data}
\item{Find relevant tools and technology to implement the designed architecture}
\item{Test the system, and improve}
\end{enumerate}
The importance of this project has two components, one personal and one related to the sports industry. The personal importance of this project has to do with the team working on it. This project has been, and will hopefully continue to be, a valuable learning opportunity. It has allowed the creators to learn more about data collection, machine learning, software projects, best practices, and engineering design. On the other hand, with respect to the sports industry, given a system that can accurately and consistently predict winning line-ups for fantasy basketball, the way in which fantasy sports are played may change. Rather than rely on intuition, or optimization techniques alone, there may be an incentive to use a machine learning approach if the lineups from such a system are higher scoring. This could not only change how fantasy competitions are played, but could be adapted to select which players should actually be played, in real life. Theoretically, a coach could use the same technique, although with a different success metric (winning a game instead of maximizing fantasy points) to figure out who to play. In both cases, machine learning could help find patterns that we did not know exist.

This report is structured as follows. First it will provide background on the fantasy basketball competitions that we targeted. Then, it will briefly summarize the requirements of the system. Following that, it will provide a detailed overview of the design of the system, and the results. Finally, it will discuss future plans, the impact on society and the environment, and how the work was divided amongst the authors.