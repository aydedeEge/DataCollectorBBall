%------------------------------------------------------------------
% Plan for Next Semester Section:
% In this section you should describe a plan for next semester. What tasks remain to be done, and what is the timeline for accomplishing these tasks? Furthermore you can describe how you plan to test your design in order to make sure it meets the desired specifications. Also outline the design decisions that are required in order to make sure that the final product can be tested.
%------------------------------------------------------------------

We plan to continue to improve on the design and implementation of this project. We think that there are many areas that can be improved, and would like to divide these into two sections: general improvements, and machine learning improvements.

\subsection{General Improvements}
There are many ways that the entire system can be improved in a general sense. For one, it would be possible to add additional features (inputs) that could increase the performance of the NN. Next, we could improve the scoring calculation function of system one, either by removing it and instead passing all of the players' stats as input to the NN, or by coming up with a more representative formula that we validate on. Next, we could factor in the standard deviation of each player in order to have a risky value associated with each generated lineup. Lineups with high-variance players are more risky, whereas with more consistent players are less risky. Finally, currently, in order to run the system on a day, it requires running three unoptimized scripts, which take in total around 15 minutes. This could easily be consolidated into running a single script that takes less time. This would make it more convenient to run. 

\subsection{Machine Learning Improvements}
This subsection will detail the different machine learning improvements that we feel could be made on our system. In particular, this involves the second stage of the system, the neural network. Currently, although we performed cross-validation on the network, we did not perform as thorough a cross-validation as would be ideal. As discussed, the system was validated on the number of hidden layers (two through four), the number of nodes per hidden layer (32, 64, 128), and the learning rate (0.01, 0.1). Not only would we like to increase the amount of values for nodes per hidden layer and learning rate that we validate on, we would also like to add other hyperparameters, including the amount of days taken for a short term score, the amount of data used (how many years), the dropout used, and the weight on the error of recent games (to make more recent games worth more).
Second, we would like to incorporate a machine learning technique to reduce the variance of our predictions. Through some research, we have found that a technique called Bootstrap Aggregation, or Bagging, would be suitable here. A bootstrapped dataset is a dataset created out of the original dataset. With an original dataset of size n, a bootstrapped dataset would be created by randomly taking n values from this original dataset with replacement. With replacement means that it is possible to take the same value multiple times. Bagging is the process of creating ``l'' bootstrapped datasets, training the network on all ``l'' of these, and then averaging the results for each prediction. This ideally will reduce the overall variance. 
